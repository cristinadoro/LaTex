\begin{frame}
\frametitle{Al doilea capitol} 
Lorem ipsum dolor sit amet, consectetur adipiscing elit. Nam volutpat felis et sem blandit, et porttitor velit tincidunt. Nam mollis vestibulum aliquet. 
\begin{equation}
\int_a^b x^2 \alpha \eta \forall x \in X, \quad \exists y \leq \epsilon
\end{equation}
Etiam sodales enim consectetur orci placerat, nec imperdiet massa egestas. Etiam varius sagittis pretium. Nunc at mattis magna.
$$|x| = \left\{\begin{array}{rl} -x, & \textrm{daca $x<0$;}\\ x, & \textrm{altfel.} \end{array} \right. $$
Quisque aliquam tellus quis scelerisque consequat. Nullam facilisis vel metus non vehicula.
$$y = \sqrt{1+\sqrt{1+\sqrt{1+ \sqrt{1+x}}}}$$
Mauris nibh leo, lobortis quis mauris ac, euismod consequat diam. Vivamus tristique risus dolor, vitae luctus sapien iaculis ac. Sed ultricies nisl leo, et ultrices purus suscipit et. In condimentum metus sed elit fermentum congue. Donec nec commodo dui. Cras vel dapibus felis, sit amet mollis sem. Aenean cursus augue imperdiet posuere aliquam. 
Nulla vel dignissim elit. Aenean volutpat lectus quis tortor lobortis, vitae volutpat tellus ultrices. Nullam eu leo quis diam lobortis porttitor consectetur non neque. Fusce dignissim dolor ut arcu rutrum, sed consectetur leo pretium.
\begin{eqnarray}
\biggl(\int_{-\infty} ^\infty e^{-x^2}\,dx\biggr)^2 & =& \int_{-\infty}^\infty \int_{-\infty}^\infty e^{-(x^2+y^2)}\,dx\,dy \nonumber \\ & =& \int_0^{2\pi}\int_0^\infty e^{-r^2}r\,dr\,d\theta \nonumber \\ & =& \int_0^{2\pi}\biggl(-{e^{-r^2}\over2} \bigg|_{r=0}^{r=\infty}\,\biggr)\,d\theta \nonumber \\ & =& \pi /\end{eqnarray}
 Quisque purus dui, consequat ut libero quis, vulputate bibendum quam. Suspendisse urna lectus, ullamcorper eleifend neque ac, aliquet placerat dui. Nam fermentum tincidunt est, viverra auctor purus. Pellentesque viverra arcu sed purus lobortis pharetra. Mauris consequat lectus id ligula porta posuere.
$$ \prod_{j\ge0}\biggl( \sum_{k\ge0}a_{jk}z^k\biggr) =\sum_{n\ge0}z^n\,\Biggl(\sum_ {\scriptstyle k_0,k_1, \ldots\ge0\atop \scriptstyle k_0+k_1+\cdots=n} a_{0k_0}a_{1k_1}\ldots\,\Biggr) $$
Suspendisse sed feugiat velit. Nunc ut sapien sit amet odio tincidunt semper. Pellentesque lobortis ipsum non aliquet ornare. Fusce rhoncus ligula id ligula aliquam rhoncus. Suspendisse iaculis orci sapien, in imperdiet lorem varius vel. Etiam tempor, justo quis varius tincidunt, lacus arcu euismod nibh, non consequat ligula augue nec leo. Phasellus pretium eleifend congue. Donec ultricies eget risus et scelerisque.
\begin{equation} f(x)=\sqrt{(x+a)(x+b)} \end{equation}
Suspendisse potenti. Lorem ipsum dolor sit amet, consectetur adipiscing elit. Ut in luctus erat. Sed arcu tellus, porta sed sollicitudin quis, tempor at libero. Morbi adipiscing dignissim tristique. In hac habitasse platea dictumst. Nulla bibendum consequat mollis.
$$ \underbrace{a+\overbrace{\sqrt{(x+a)(x+b)}+\cdots+y}^{24}+z}_{26} $$
Praesent eget ultricies neque. Vivamus vel mauris at nulla volutpat varius. Aliquam mattis, metus non fermentum posuere, urna mauris fringilla ante, non convallis nisl diam non lorem. Mauris et nisl tristique, tempus nunc ac, hendrerit sem. Nulla quis tortor lorem. Nullam tristique mollis diam, quis elementum tortor rhoncus in. Morbi sem sapien, malesuada et nisl cursus, dictum tincidunt est.
\end{frame}
